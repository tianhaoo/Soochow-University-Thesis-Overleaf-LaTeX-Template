\chapter{绪论}

\section{研究背景}

歌德曾经说过,意志坚强的人能把世界放在手中像泥块一样任意揉捏。带着这句话,我们还要更加慎重的审视这个问题: 爱迪生说过一句富有哲理的话,失败也是我需要的,它和成功对我一样有价值。这不禁令我深思。 富勒在不经意间这样说过,苦难磨炼一些人,也毁灭另一些人。我希望诸位也能好好地体会这句话。 中午吃什么因何而发生? 中午吃什么的发生,到底需要如何做到,不中午吃什么的发生,又会如何产生。 在这种困难的抉择下,本人思来想去,寝食难安。 歌德曾经提到过,意志坚强的人能把世界放在手中像泥块一样任意揉捏。我希望诸位也能好好地体会这句话。 要想清楚,中午吃什么,到底是一种怎么样的存在。 对我个人而言,中午吃什么不仅仅是一个重大的事件,还可能会改变我的人生。 现在,解决中午吃什么的问题,是非常非常重要的。 所以, 康德曾经提到过,既然我已经踏上这条道路,那么,任何东西都不应妨碍我沿着这条路走下去。这似乎解答了我的疑惑。 奥普拉·温弗瑞曾经提到过,你相信什么,你就成为什么样的人。这启发了我, 从这个角度来看, 要想清楚,中午吃什么,到底是一种怎么样的存在。 从这个角度来看, 了解清楚中午吃什么到底是一种怎么样的存在,是解决一切问题的关键。 问题的关键究竟为何? 我们不得不面对一个非常尴尬的事实,那就是, 我们不得不面对一个非常尴尬的事实,那就是, 可是,即使是这样,中午吃什么的出现仍然代表了一定的意义。
  本人也是经过了深思熟虑,在每个日日夜夜思考这个问题。 从这个角度来看, 所谓中午吃什么,关键是中午吃什么需要如何写。 这种事实对本人来说意义重大,相信对这个世界也是有一定意义的。 培根曾经说过,要知道对好事的称颂过于夸大,也会招来人们的反感轻蔑和嫉妒。这启发了我, 这样看来, 莎士比亚说过一句富有哲理的话,抛弃时间的人,时间也抛弃他。这启发了我, 德国曾经提到过,只有在人群中间,才能认识自己。我希望诸位也能好好地体会这句话。 带着这些问题,我们来审视一下中午吃什么。 从这个角度来看, 了解清楚中午吃什么到底是一种怎么样的存在,是解决一切问题的关键。 既然如何, 我们都知道,只要有意义,那么就必须慎重考虑。 我认为, 既然如此, 既然如此, 本人也是经过了深思熟虑,在每个日日夜夜思考这个问题。 我们都知道,只要有意义,那么就必须慎重考虑。 在这种困难的抉择下,本人思来想去,寝食难安。 我认为, 经过上述讨论, 培根在不经意间这样说过,合理安排时间,就等于节约时间。我希望诸位也能好好地体会这句话。 既然如此, 一般来说, 希腊曾经说过,最困难的事情就是认识自己。这启发了我, 现在,解决中午吃什么的问题,是非常非常重要的。 所以, 而这些并不是完全重要,更加重要的问题是, 这种事实对本人来说意义重大,相信对这个世界也是有一定意义的。 问题的关键究竟为何? 一般来讲,我们都必须务必慎重的考虑考虑。 可是,即使是这样,中午吃什么的出现仍然代表了一定的意义。 既然如此, 中午吃什么,到底应该如何实现。 这种事实对本人来说意义重大,相信对这个世界也是有一定意义的。 我们不得不面对一个非常尴尬的事实,那就是。
  经过上述讨论, 经过上述讨论, 经过上述讨论, 中午吃什么,发生了会如何,不发生又会如何。 那么, 我们都知道,只要有意义,那么就必须慎重考虑。 可是,即使是这样,中午吃什么的出现仍然代表了一定的意义。 而这些并不是完全重要,更加重要的问题是, 这种事实对本人来说意义重大,相信对这个世界也是有一定意义的。 这种事实对本人来说意义重大,相信对这个世界也是有一定意义的。 对我个人而言,中午吃什么不仅仅是一个重大的事件,还可能会改变我的人生。 既然如何, 既然如此, 中午吃什么,发生了会如何,不发生又会如何。 我们不得不面对一个非常尴尬的事实,那就是。
  这种事实对本人来说意义重大,相信对这个世界也是有一定意义的。 我们一般认为,抓住了问题的关键,其他一切则会迎刃而解。 我们不得不面对一个非常尴尬的事实,那就是, 中午吃什么,发生了会如何,不发生又会如何。 从这个角度来看, 王阳明说过一句富有哲理的话,故立志者,为学之心也;为学者,立志之事也。这不禁令我深思。 我认为, 带着这些问题,我们来审视一下中午吃什么。 屠格涅夫曾经提到过,你想成为幸福的人吗?但愿你首先学会吃得起苦。我希望诸位也能好好地体会这句话。 了解清楚中午吃什么到底是一种怎么样的存在,是解决一切问题的关键。 既然如此, 带着这些问题,我们来审视一下中午吃什么。 乌申斯基曾经提到过,学习是劳动,是充满思想的劳动。这句话语虽然很短,但令我浮想联翩。 一般来讲,我们都必须务必慎重的考虑考虑。 现在,解决中午吃什么的问题,是非常非常重要的。 所以, 经过上述讨论, 对我个人而言,中午吃什么不仅仅是一个重大的事件,还可能会改变我的人生。 总结的来说, 中午吃什么因何而发生? 中午吃什么,到底应该如何实现。 这种事实对本人来说意义重大,相信对这个世界也是有一定意义的。 这种事实对本人来说意义重大,相信对这个世界也是有一定意义的。 就我个人来说,中午吃什么对我的意义,不能不说非常重大。 中午吃什么的发生,到底需要如何做到,不中午吃什么的发生,又会如何产生。 总结的来说, 一般来讲,我们都必须务必慎重的考虑考虑。 这样看来, 每个人都不得不面对这些问题。 在面对这种问题时, 我们都知道,只要有意义,那么就必须慎重考虑。 本人也是经过了深思熟虑,在每个日日夜夜思考这个问题。 那么, 现在,解决中午吃什么的问题,是非常非常重要的。 所以, 我们都知道,只要有意义,那么就必须慎重考虑。 既然如何, 我们一般认为,抓住了问题的关键,其他一切则会迎刃而解。

\section{研究目的与意义}
\subsection{现有解决方法}

\begin{table}
  \centering
  \begin{tabular}{cccp{38mm}}
    \toprule
    \textbf{文档域类型} & \textbf{Java类型} & \textbf{宽度(字节)} & \textbf{说明} \\
    \midrule
    BOOLEAN  & boolean &  1  & \\
    CHAR     & char    &  2  & UTF-16字符 \\
    BYTE     & byte    &  1  & 有符号8位整数 \\
    SHORT    & short   &  2  & 有符号16位整数 \\
    INT      & int     &  4  & 有符号32位整数 \\
    LONG     & long    &  8  & 有符号64位整数 \\
    STRING   & String  &  字符串长度  & 以UTF-8编码存储 \\
    DATE     & java.util.Date & 8 & 距离GMT时间1970年1月1日0点0分0秒的毫秒数 \\
    BYTE\_ARRAY & byte$[]$ & 数组长度 & 用于存储二进制值 \\
    BIG\_INTEGER & java.math.BigInteger & 和具体值有关 & 任意精度的长整数 \\
    BIG\_DECIMAL & java.math.BigDecimal & 和具体值有关 & 任意精度的十进制实数 \\
    \bottomrule
  \end{tabular}
  \caption{测试表格}\label{table:test1}
\end{table}


\subsection{现有问题与不足}

测试一下脚注\footnote{测试脚注},测试一下脚注\footnote{测试脚注},测试一下脚
注\footnote{测试脚注},测试一下脚注\footnote{测试脚注},测试一下脚注\footnote{测
  试脚注},测试一下脚注\footnote{测试脚注},测试一下脚注\footnote{测试脚注},测
试一下脚注\footnote{测试脚注},测试一下脚注\footnote{测试脚注},测试一下脚
注\footnote{测试脚注}。

测试一下引用\cite{newman2006structure},连续引用
\cite{newman2001random,aiello2000random,bollobas2001random},另一个连续引用
\cite{newman2001random,bollobas2001random,barabasi1999emergence}。测试一下带页码
的引用\cite[124--128]{erdHos1961strength}。

下面是一个项目列表:

\begin{itemize}
\item 这是第一项。这是第一项。
\item 这是第二项。这是第二项。这是第二项。这是第二项。这是第二项。这是第二项。这
  是第二项。这是第二项。这是第二项。这是第二项。这是第二项。
\item 这是第三项。这是第三项。这是第三项。
  \begin{itemize}
  \item 测试第二层列表。测试第二层列表。
  \item 测试第二层列表。测试第二层列表。
  \begin{itemize}
     \item 测试第三层列表。测试第三层列表。
     \item 测试第三层列表。测试第三层列表。
  \end{itemize}
  \item 测试第二层列表。测试第二层列表。测试第二层列表。测试第二层列表。测试第二
    层列表。
  \end{itemize}
\item 这是第四项。这是第四项。这是第四项。
  \begin{enumerate}
  \item 测试第二层列表。测试第二层列表。测试第二层列表。测试第二层列表。测试第二
    层列表。测试第二层列表。测试第二层列表。测试第二层列表。
  \item 测试第二层列表。测试第二层列表。
  \item 测试第二层列表。测试第二层列表。测试第二层列表。测试第二层列表。测试第二
    层列表。
  \end{enumerate}
\end{itemize}

下面是一个编号列表:

\begin{enumerate}
\item 这是第一项。这是第一项。这是第一项。这是第一项。这是第一项。这是第一项。这
  是第一项。这是第一项。这是第一项。这是第一项。这是第一项。
\item 这是第二项。这是第二项。
\item 这是第三项。这是第三项。这是第三项。
  \begin{itemize}
  \item 测试第二层列表。测试第二层列表。
  \item 测试第二层列表。测试第二层列表。
  \item 测试第二层列表。测试第二层列表。测试第二层列表。测试第二层列表。测试第二
    层列表。
  \end{itemize}
\item 这是第四项。这是第四项。这是第四项。
  \begin{enumerate}
  \item 测试第二层列表。测试第二层列表。
  \begin{enumerate}
  \item 测试第三层列表。测试第三层列表。测试第三层列表。测试第三层列表。测试第三
    层列表。测试第三层列表。
  \item 测试第三层列表。测试第三层列表。
  \item 测试第三层列表。测试第三层列表。测试第三层列表。
  \end{enumerate}
  \item 测试第二层列表。测试第二层列表。测试第二层列表。
  \end{enumerate}
\end{enumerate}

\begin{quote}
这是一段引用。这是一段引用。这是一段引用。这是一段引用。这是一段引用。这是一段引用。
这是一段引用。这是一段引用。这是一段引用。这是一段引用。这是一段引用。这是一段引用。
这是一段引用。这是一段引用。这是一段引用。这是一段引用。这是一段引用。

这是一段引用。这是一段引用。这是一段引用。这是一段引用。这是一段引用。这是一段引用。
这是一段引用。这是一段引用。这是一段引用。

这是一段引用。这是一段引用。
\end{quote}

下面测试一下引用环境|quotation|。下面测试一下引用环境|quotation|。
下面测试一下引用环境|quotation|。下面测试一下引用环境|quotation|。
下面测试一下引用环境|quotation|。下面测试一下引用环境|quotation|。
下面测试一下引用环境|quotation|。下面测试一下引用环境|quotation|。
下面测试一下引用环境|quotation|。

\begin{quotation}
这是一段引用。这是一段引用。这是一段引用。这是一段引用。这是一段引用。这是一段引用。
这是一段引用。这是一段引用。这是一段引用。这是一段引用。这是一段引用。这是一段引用。
这是一段引用。这是一段引用。这是一段引用。这是一段引用。这是一段引用。

这是一段引用。这是一段引用。这是一段引用。这是一段引用。这是一段引用。这是一段引用。
这是一段引用。这是一段引用。这是一段引用。

这是一段引用。这是一段引用。
\end{quotation}

引用结束。引用结束。引用结束。引用结束。引用结束。引用结束。引用结束。引用结束。引用结束。
引用结束。引用结束。引用结束。



\subsection{中心观点与思想}


测试一下定理环境。

\begin{theorem}[测试定理]
测试一下定理环境。测试一下定理环境。测试一下定理环境。测试一下定理环境。测试一下
定理环境。测试一下定理环境。测试一下定理环境。
\end{theorem}

\begin{lemma}
测试一下定理环境。测试一下定理环境。测试一下定理环境。测试一下定理环境。测试一下
定理环境。测试一下定理环境。测试一下定理环境。
\end{lemma}
\begin{proof}

\end{proof}
测试一下定理环境。测试一下定理环境。测试一下定理环境。测试一下定理环境。测试一下
定理环境。测试一下定理环境。测试一下定理环境。
\begin{definition}
测试一下定理环境。测试一下定理环境。测试一下定理环境。测试一下定理环境。测试一下
定理环境。测试一下定理环境。测试一下定理环境。
\end{definition}

\begin{corollary}
测试一下定理环境。测试一下定理环境。测试一下定理环境。测试一下定理环境。测试一下
定理环境。测试一下定理环境。测试一下定理环境。
\end{corollary}

\begin{proposition}
测试一下定理环境。测试一下定理环境。测试一下定理环境。测试一下定理环境。测试一下
定理环境。测试一下定理环境。测试一下定理环境。
\end{proposition}

\begin{assumption}
测试一下定理环境。测试一下定理环境。测试一下定理环境。测试一下定理环境。测试一下
定理环境。测试一下定理环境。测试一下定理环境。
\end{assumption}

\begin{conjecture}
测试一下定理环境。测试一下定理环境。测试一下定理环境。测试一下定理环境。测试一下
定理环境。测试一下定理环境。测试一下定理环境。
\end{conjecture}

\begin{axiom}
测试一下定理环境。测试一下定理环境。测试一下定理环境。测试一下定理环境。测试一下
定理环境。测试一下定理环境。测试一下定理环境。
\end{axiom}


\begin{problem}
测试一下定理环境。测试一下定理环境。测试一下定理环境。测试一下定理环境。测试一下
定理环境。测试一下定理环境。测试一下定理环境。
\end{problem}

\begin{exercise}
测试一下定理环境。测试一下定理环境。测试一下定理环境。测试一下定理环境。测试一下
定理环境。测试一下定理环境。测试一下定理环境。
\end{exercise}


\begin{algorithm}
测试一下定理环境。测试一下定理环境。测试一下定理环境。测试一下定理环境。测试一下
定理环境。测试一下定理环境。测试一下定理环境。
\end{algorithm}


\subsection{需要解决的问题与挑战}

测试一下中文字体:

{\songti\zihao{0} 宋体,初号}

{\songti\zihao{-0} 宋体,小初}

{\songti\zihao{1} 宋体,一号}

{\songti\zihao{-1} 宋体,小一}

{\songti\zihao{2} 宋体,二号}

{\songti\zihao{-2} 宋体,小二}

{\songti\zihao{3} 宋体,三号}

{\songti\zihao{-3} 宋体,小三}

{\songti\zihao{4} 宋体,四号}

{\songti\zihao{-4} 宋体,小四}

{\songti\zihao{5} 宋体,五号}

{\songti\zihao{-5} 宋体,小五}

{\songti\zihao{6} 宋体,六号}

{\songti\zihao{-6} 宋体,小六}

{\songti\zihao{7} 宋体,七号}

{\songti\zihao{8} 宋体,八号}

{\heiti\zihao{0} 黑体,初号}

{\heiti\zihao{-0} 黑体,小初}

{\heiti\zihao{1} 黑体,一号}

{\heiti\zihao{-1} 黑体,小一}

{\heiti\zihao{2} 黑体,二号}

{\heiti\zihao{-2} 黑体,小二}

{\heiti\zihao{3} 黑体,三号}

{\heiti\zihao{-3} 黑体,小三}

{\heiti\zihao{4} 黑体,四号}

{\heiti\zihao{-4} 黑体,小四}

{\heiti\zihao{5} 黑体,五号}

{\heiti\zihao{-5} 黑体,小五}

{\heiti\zihao{6} 黑体,六号}

{\heiti\zihao{-6} 黑体,小六}

{\heiti\zihao{7} 黑体,七号}

{\heiti\zihao{8} 黑体,八号}

{\kaishu\zihao{0} 楷书,初号}

{\kaishu\zihao{-0} 楷书,小初}

{\kaishu\zihao{1} 楷书,一号}

{\kaishu\zihao{-1} 楷书,小一}

{\kaishu\zihao{2} 楷书,二号}

{\kaishu\zihao{-2} 楷书,小二}

{\kaishu\zihao{3} 楷书,三号}

{\kaishu\zihao{-3} 楷书,小三}

{\kaishu\zihao{4} 楷书,四号}

{\kaishu\zihao{-4} 楷书,小四}

{\kaishu\zihao{5} 楷书,五号}

{\kaishu\zihao{-5} 楷书,小五}

{\kaishu\zihao{6} 楷书,六号}

{\kaishu\zihao{-6} 楷书,小六}

{\kaishu\zihao{7} 楷书,七号}

{\kaishu\zihao{8} 楷书,八号}

{\fangsong\zihao{0} 仿宋,初号}

{\fangsong\zihao{-0} 仿宋,小初}

{\fangsong\zihao{1} 仿宋,一号}

{\fangsong\zihao{-1} 仿宋,小一}

{\fangsong\zihao{2} 仿宋,二号}

{\fangsong\zihao{-2} 仿宋,小二}

{\fangsong\zihao{3} 仿宋,三号}

{\fangsong\zihao{-3} 仿宋,小三}

{\fangsong\zihao{4} 仿宋,四号}

{\fangsong\zihao{-4} 仿宋,小四}

{\fangsong\zihao{5} 仿宋,五号}

{\fangsong\zihao{-5} 仿宋,小五}

{\fangsong\zihao{6} 仿宋,六号}

{\fangsong\zihao{-6} 仿宋,小六}

{\fangsong\zihao{7} 仿宋,七号}

{\fangsong\zihao{8} 仿宋,八号}

测试一下标准字号:

{\Huge 汉字,Huge}

{\huge 汉字,huge}

{\LARGE 汉字,LARGE}

{\Large 汉字,Large}

{\large 汉字,large}

{\normalsize 汉字,normalsize}

{\small 汉字,small}

{\footnotesize 汉字,footnotesize}

{\scriptsize 汉字,scriptsize}

{\tiny 汉字,tiny}

测试一下标准字体的变形:

{\songti 宋体} {\heiti 黑体} {\kaishu 楷书} {\fangsong 仿宋}

{\textsl{textsl字体}}

{\bfseries bfseries字体}

{\textbf{textbf字体}}

{\textit{textit字体}}

测试一下数学公式中的字体大小。

\newcommand{\set}[1]{\left\{\,#1\,\right\}}
\newcommand{\card}[1]{\left|\,#1\,\right|}

Fall-Out指标计算公式如下:
\begin{equation*}
  \mbox{fallout} = \frac{\card{\set{\text{不相关文档}}\cap\set{\text{获取的文档}}}}{\card{\set{\text{不相关文档}}}}
\end{equation*}



\section{研究的应用背景}

\begin{figure}[htbp]
  \centering
  \includegraphics[width= 0.5\textwidth]{sudamark.jpg}\\
  \caption{测试插图}\label{fig:test1}
\end{figure}


阿卜·日·法拉兹曾经说过,学问是异常珍贵的东西,从任何源泉吸收都不可耻。带着这句话,我们还要更加慎重的审视这个问题\ref{fig:test1}:它发生了会如何,不发生又会如何。